%!TEX root = ../Satzung.tex
\chapter{Allgemeine Bestimmungen}
\label{sec:Allg_Bestimmung}


\section{Name und Sitz}
	\begin{absätze}
		\item Der Verein führt den Namen \textit{\Vereinsname{}} bzw. in der Kurzform \textit{\VereinsnameKurz{}}.
		\item Der Verein hat seinen Sitz in \Stadt{}.
		\item Er soll in das Vereinsregister eingetragen werden und führt danach den Zusatz ''e.V.''.
	\end{absätze}

\section{Zweck und Aufgaben des Vereins}
	\begin{absätze}
		\item Der Verein verfolgt ausschließlich und unmittelbar gemeinnützige Zwecke im Sinne des Abschnitts "`Steuerbegünstigte Zwecke"' der Abgabenordnung.
		\item Zweck des Vereins ist die Förderung der Erziehung, Volks- und Berufsbildung einschließlich der Studierendenhilfe.
		\item Der Satzungszweck wird insbesondere verwirklicht durch:
		\begin{sätze}
			\item die Förderung der Interessen der Studierenden am Fachbereichs \Fach{} der \Uni{},
			\item die ideelle und finanzielle Unterstützung des Fachschaftsrates am Fachbereichs \Fach{} gemäß \VerweisFachschaftsrat{} sowie
			\item die Förderung von Lehre und Forschung am Fachbereichs \Fach{} der \Uni{}.
		\end{sätze}
		\item Die inhaltliche Arbeit des Vereins wird durch den Vorstand und die Mitglieder getragen. Über finanzielle Förderungen entscheidet der Vorstand. Näheres kann in einer Ordnung geregelt werden.
	\end{absätze}

\section{Selbstlosigkeit}
	\begin{absätze}
		\item Der Verein ist selbstlos tätig. Er verfolgt nicht in erster Linie eigenwirtschaftliche Zwecke.
		\item Der Verein verwendet seine Mittel nur für die satzungsgemäßen Zwecke.
		\item Die Mitglieder erhalten keine Gewinnanteile und in ihrer Eigenschaft als Mitglieder auch keine sonstigen Zuwendungen aus den Mitteln des Vereins. Sie haben bei ihrem Ausscheiden aus dem Verein oder bei dessen Auflösung keine Ansprüche auf Vereinsvermögen.
		\item Der Verein setzt keine Mittel zur Unterstützung politischer Parteien oder religiöser Gruppen ein.
		\item Der Verein begünstigt keine Person durch Ausgaben, die dem satzungsgemäßen Zweck widersprechen oder durch unverhältnismäßig hohe Vergütungen.
		\item Bei Auflösung des Vereins oder bei Wegfall steuerbegünstigter Zwecke fällt das Vermögen an die \Uni{}, die es unmittelbar und ausschließlich für gemeinnützige oder mildtätige steuerbegünstigte, dieser Satzung entsprechende Zwecke zu verwenden hat.
	\end{absätze}
