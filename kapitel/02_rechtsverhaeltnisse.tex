%!TEX root = ../Satzung.tex
\chapter{Rechtsverhältnisse des Vereins und seiner Mitglieder}

\section{Mitgliedschaft}\label{sec:mitgliedschaft}
\begin{absätze}
    \item Ordentliches Mitglied können alle natürlichen Personen werden, die Teil des \Fachbereich{} der \Uni{} sind.
    \item Fördermitglied können alle Personen werden, die die Ziele des Vereins unterstützen.
\end{absätze}

\section{Erwerb der Mitgliedschaft}\label{sec:erwerb_der_mitgliedschaft}
\begin{absätze}
    \item Die Mitgliedschaft wird durch Annahme eines an den Vorstand gerichteten schriftlichen Antrags erworben.
    \item Der Antrag kann abgelehnt werden, falls die betreffende Person schon einmal Mitglied des Vereins gewesen ist. In diesem Fall kann einer Mitgliederversammlung persönlich der Widerspruch vorgebracht werden, die dann über die Aufnahme entscheidet.
    \item Die Mitgliedschaft kann den Mitgliedern des Fachschaftsrates am \Fachbereich{} gemäß \VerweisFachschaftsrat{} nicht verwehrt werden.
\end{absätze}

\section{Ende der Mitgliedschaft}\label{sec:ende_der_mitgliedschaft}
\begin{absätze}
    \item Die Mitgliedschaft erlischt
    \begin{sätze}
        \item wenn das Mitglied schriftlich gegenüber dem Vorstand seinen Austritt erklärt,
        \item wenn das Mitglied mit der Zahlung der Mitgliedsbeiträge zwei Jahre im Rückstand ist und der Vorstand daraufhin das Ende der Mitgliedschaft feststellt oder
        \item mit dem Tod des Mitglieds.
    \end{sätze}
    \item Die ordentliche Mitgliedschaft geht in eine Fördermitgliedschaft über, wenn
    \begin{sätze}
        \item das Mitglied in einem Zeitraum von zwei Jahren zu keiner Mitgliederversammlung erschienen ist oder
        \item das Mitglied nicht mehr die Bedingungen des \ref{sec:mitgliedschaft} Abs. 1 erfüllt. Das Mitglied muss die Nichterfüllung dem Vorstand anzeigen.
    \end{sätze}
    \item Ein Mitglied kann aus dem Verein ausgeschlossen werden, wenn
    \begin{sätze}
        \item das Mitglied gegen die Satzung verstößt,
        \item das Vereinsansehen schädigt oder
        \item den Verein vorsätzlich oder fahrlässig wirtschaftlich in erheblichem Ausmaß schädigt
    \end{sätze}
    und die Mitgliederversammlung daraufhin mit einer Mehrheit von mindestens drei Vierteln den Ausschluss beschließt. Bei der Abstimmung ist das betroffene Mitglied nicht stimmberechtigt.
	\end{absätze}

\section{Mitgliedsbeiträge}
	\begin{absätze}
		\item Von den ordentlichen Mitgliedern und den Fördermitgliedern kann ein jährlicher, ggf. nach Mitgliedsform differenzierter Beitrag erhoben werden, welcher unabhängig vom Beitrittstermin jeweils für das Kalenderjahr erhoben wird.
		\item Über die Höhe der Beiträge entscheidet die Mitgliederversammlung.
	\end{absätze}

\section{Organe des Vereins}
    Organe des Vereins sind:
	\begin{compactenum}
		\item Mitgliederversammlung und
		\item der Vorstand.
	\end{compactenum}
