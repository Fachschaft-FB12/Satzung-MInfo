%!TEX root = ../Satzung.tex
\chapter{Mitgliederversammlung}

\section{Zuständigkeiten}\label{sec:mv-zustaendigkeit}
\begin{absätze}
	\item Oberstes Organ des Vereins ist die Mitgliederversammlung, die satzungsgemäß einberufene Versammlung der Vereinsmitglieder.
    \item Der Mitgliederversammlung obliegt die Ordnung der Angelegenheiten des Vereins durch Beschlussfassung. Diese sind unter anderem:
    \begin{sätze}
        \item Wahl und Abwahl des Vorstandes,
        \item Entlastung des Vorstands,
        \item Wahl der Kassenprüfer*innen des Vereins,
        \item Entscheidung über die Erhebung von Beiträgen und Verabschiedung einer Beitragsordnung,
        \item Annahme von durch den Vorstand abgelehnten Mitgliedsanträgen gemäß \ref{sec:erwerb_der_mitgliedschaft} Abs. 2,
        \item Entscheidung über Ausschluss von Mitgliedern gemäß \ref{sec:ende_der_mitgliedschaft} Abs. 3,
        \item Änderungen der Vereinssatzung sowie
        \item Auflösung des Vereins.
    \end{sätze}
\end{absätze}

\section{Turnus, Öffentlichkeit}
\begin{absätze}
    \item Die ordentliche Mitgliederversammlung findet jährlich statt, in der Regel im vierten Quartal.
    \item Jede Mitgliederversammlung findet öffentlich statt.
\end{absätze}

\section{Einberufung}\label{sec:einberufung}
Die Einladung zur Mitgliederversammlung wird den Mitgliedern spätestens vierzehn Tage vor der Mitgliederversammlung unter Angabe der Tagesordnung an die von ihnen angegebene E-Mail-Adresse vom Vorstand zugesandt.

\section{Tagesordnung}
\begin{absätze}
    \item Die Mitglieder können bis zum Beginn einer Mitgliederversammlung die Ergänzung der Tagesordnung bei dem Vorstand beantragen. Über die Annahme der Anträge entscheidet die Mitgliederversammlung. Nicht Gegenstand eines Ergänzungsantrags können sein:
    \begin{sätze}
        \item Auflösung des Vereins,
        \item Änderung der Satzung sowie
        \item Ausschluss eines oder mehrerer Mitglieder.
    \end{sätze}
    \item Einzelne Tagesordnungspunkte können auf Beschluss der Mitgliederversammlung unter Ausschluss der Öffentlichkeit behandelt werden.
    \item Die Tagesordnung jeder ordentlichen Mitgliederversammlung muss mindestens die folgenden Tagesordnungspunkte beinhalten:
    \begin{sätze}
        \item Beschluss der Tagesordnung,
        \item Feststellung der Beschlussfähigkeit,
        \item Annahme des Protokolls der letzten Mitgliederversammlung,
        \item Geschäftsbericht des Vorstandes,
        \item Bericht der Kassenprüfer*innen sowie
        \item Wahl und Entlastung des Vorstandes.
    \end{sätze}
\end{absätze}

\section{Beschlussfähigkeit}
\begin{absätze}
    \item Eine ordentliche oder außerordentliche Mitgliederversammlung ist beschlussfähig, wenn sie gemäß \ref{sec:einberufung} Abs. 2 einberufen wurde und mindestens die Hälfte der ordentlichen Mitglieder anwesend sind.
    \item Falls die Mitgliederversammlung ihre eigene Beschlussunfähigkeit feststellt, muss der Vorstand binnen sieben Tagen zu einer weiteren Mitgliederversammlung mit gleicher Tagesordnung einladen. Die Einladung muss unter Berücksichtigung des \ref{sec:einberufung} erfolgen. Diese zweite Mitgliederversammlung ist unabhängig von der Zahl der anwesenden, ordentlichen Mitglieder beschlussfähig, sofern dies auf der Einladung angegeben ist.
\end{absätze}

\section{Leitung der Mitgliederversammlung}
Die Leitung der Mitgliederversammlung wird vom Vorstand festgelegt.

\section{Stimmrecht}
\begin{absätze}
    \item Jedes anwesende, ordentliche Mitglied ist in der Mitgliederversammlung einfach stimmberechtigt.
    \item Die Personen im Vorstand des Vereins sind einfach stimmberechtigt.
    \item Fördermitglieder sind nicht stimmberechtigt.
\end{absätze}

\section{Beschlussfassung}
\begin{absätze}
    \item Bei Beschlussfassungen kann jedes anwesende, stimmberechtigte Mitglied mit \textit{Ja}, \textit{Nein} oder \textit{Enthaltung} stimmen. Zur Beschlussfassung genügt, falls nicht anders geregelt, eine Mehrheit der \textit{Ja}- über den \textit{Nein}-Stimmen. Bei Stimmengleichheit ist der Antrag abgelehnt.
    % \item Bei der Wahl des Vorstandes oder der Kassenprüfer*innen sind, falls mehr Kandidat*innen zur Wahl stehen, als es Ämter zu besetzen gilt, die drei bzw. zwei Kandidat*innen, die die meisten Stimmen auf sich vereinigen können, gewählt. Jedes abstimmende Mitglied hat so viele Stimmen, wie Ämter zu besetzen sind.
    \item Bei der Wahl des Vorstandes oder der Kassenprüfer*innen gilt: Jedes anwesende, stimmberechtigte Mitglied hat so viele Stimmen, wie Ämter zu besetzen sind. Stehen bei einer Wahl mehr Kandidat*innen als Ämter zur Wahl, sind die Kandidat*innen (in der Anzahl der zu wählenden Ämter) gewählt, die die meisten Stimmen auf sich vereinigen. Die Stimmen eines Mitgliedes können nicht kumuliert werden. Bei Stimmengleichheit erfolgt zwischen den stimmgleichen Kandidat*innen eine Stichwahl. 
    \item Abstimmungen erfolgen stets geheim. Davon kann abgewichen werden, falls kein stimmberechtigtes Mitglied Einspruch erhebt, jedoch niemals, wenn Gegenstand der Abstimmung die Besetzung eines Amtes ist.
    \item Das Nähere kann eine Geschäftsordnung regeln.
\end{absätze}

\section{Protokoll}
\begin{absätze}
    \item Ein vom Vorstand benanntes Vereinsmitglied fertigt ein Protokoll der Mitgliederversammlung an.
    \item Das Protokoll gibt Aufschluss über die Ergebnisse von Abstimmungen.
    \item Das Protokoll ist binnen sieben Tagen nach der Mitgliederversammlung den Mitgliedern bekannt zu machen.
    % \item Den Vereinsmitgliedern bleiben dreißig Tage nach der Bekanntmachung des Protokolls zum Stellen von Änderungsanträgen, über die der Vorstand entscheidet. Wird das Protokoll dadurch nachträglich geändert, muss das Protokoll erneut entsprechend obiger Bestimmungen bekannt gemacht werden.
    \item Anträge zur Änderung des Protokolls werden durch die Mitgliederversammlung gemäß \ref{sec:mv-zustaendigkeit} beschlossen.
\end{absätze}

\section{Außerordentliche Mitgliederversammlung}
\begin{absätze}
    \item Der Vorstand veranstaltet eine außerordentliche Mitgliederversammlung, wenn
    \begin{sätze}
        \item die Belange des Vereins es erfordern,
        \item die Kassenprüfer*innen es für erforderlich halten oder
        \item mindestens ein Drittel der ordentlichen Mitglieder gemeinsam unter Angabe der Tagesordnung bei dem Vorstand eine solche beantragen.
	\end{sätze}
\item Die außerordentliche Mitgliederversammlung findet zum frühsten möglichen Zeitpunkt statt. Die in \ref{sec:einberufung} bestimmten Vorschriften zur Einberufung sind davon unberührt.
\end{absätze}
