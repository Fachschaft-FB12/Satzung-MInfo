%!TEX root = ../Satzung.tex
\chapter{Vorstand}

\section{Zusammensetzung des Vorstandes}\label{cha:vorstand}
\begin{absätze}
    \item Der Vorstand besteht aus drei gleichberechtigten Personen.
    \item Zusätzlich können bis zu vier Beisitzer*innen gewählt werden.
    \item Eines der Vorstandsmitglieder gemäß Abs. 1 muss dabei zusätzlich das Amt des*der Kassenbeauftragten bekleiden und ist mit der Führung der Kasse beauftragt.        \end{absätze}

\section{Rechtsvertretung des Vereins}
\begin{absätze}
    \item Vorstand im Sinne des § 26 BGB sind die Vorstandsmitglieder gemäß \ref{cha:vorstand} Abs. 1. Sie vertreten den Verein gemeinsam.
\end{absätze}

\section{Ehrenamtlichkeit des Vorstandes}
Die Vorstandsmitglieder sind ehrenamtlich tätig. Entstehende finanzielle Aufwendungen können ihnen vom Verein erstattet werden.
    
\section{Aufgaben des Vorstandes}
\begin{absätze}
    \item Der Vorstand führt die Geschäfte des Vereins im Rahmen dieser Satzung.
    \item Der Vorstand führt die Beschlüsse der Mitgliederversammlung aus.
    \item Der Vorstand hat auf jeder ordentlichen Mitgliederversammlung Rechenschaft über die seit der letzten Mitgliederversammlung geleistete Arbeit abzulegen.
\end{absätze}

\section{Bestellung und Abberufung des Vorstandes}
\begin{absätze}
    \item Der Vorstand wird von der Mitgliederversammlung auf die Dauer eines Jahres gewählt. Dabei wird auch die Rolle des*der Kassenbeauftragten festgelegt. Wiederwahl ist zulässig.
    \item Jedes Mitglied des Vorstands muss zum Zeitpunkt der Wahl ordentliches Mitglied des Vereins sein.
    \item Nach Ablauf der Amtszeit bleiben die Mitglieder des Vorstands bis zu einer Neuwahl des Vorstands im Amt.
    \item Eine vorgezogene Neuwahl des Vorstands kann von der Mitgliederversammlung mit einer absoluten Mehrheit beschlossen werden.
\end{absätze}

\section{Vorstandssitzungen}
\begin{absätze}
    \item Vorstandssitzungen finden nach Bedarf statt und werden grundsätzlich einvernehmlich einberufen. In Ausnahmefällen ist eine Vorstandssitzung aber auch dann einzuberufen, wenn dies von mindestens der Hälfte der Mitglieder des Vorstands beantragt wird.
    \item Der Vorstand ist beschlussfähig, wenn mindestens die Hälfte der Vorstandsmitglieder anwesend sind. Die Beschlüsse des Vorstands werden mit Stimmenmehrheit der anwesenden Vorstandsmitglieder gefasst.
    \item Das Nähere kann eine Geschäftsordnung regeln.
\end{absätze}
