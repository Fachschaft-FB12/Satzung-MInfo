\documentclass[
    a4paper,
    twoside,
]{scrreprt}

\usepackage[a4paper, right=15mm, left=15mm, top=15mm, bottom=25mm]{geometry}
\usepackage[ngerman]{babel}
\usepackage[utf8]{inputenc}
\usepackage[T1]{fontenc}
\usepackage{hyperref}
\usepackage{libertine}


% Anpassung der Überschriften an Jura
\renewcommand{\thechapter}{\Roman{chapter}}
\renewcommand{\thesection}{§ \arabic{section}}

% Globale section Nummerierung
\usepackage{remreset}
\makeatletter
  \@removefromreset{section}{chapter}
\makeatother

% Definition der Sätzen und Absätzen
\usepackage{paralist}
\newenvironment{absätze} {\begin{enumerate}[(1)]\setlength{\itemsep}{-5pt}}{\end{enumerate}}
\newenvironment{sätze}   {\begin{enumerate}[(a)]\setlength{\itemsep}{-5pt}}{\end{enumerate}}

% Definitionen: Uni
\newcommand{\Fach}{Mathematik und Informatik}
\newcommand{\Fachbereich}{Fachbereich \Fach{}}
\newcommand{\Uni}{Philipps-Universität Marburg}
\newcommand{\Stadt}{Marburg}
\newcommand{\VerweisFachschaftsrat}{Artikel 32 ff. der Satzung der Student*innenschaft der \Uni{}}

% Definitionen: Verein
\newcommand{\Vereinsname}{Förderverein der Fachschaft \Fach{} an der \Uni{}}
\newcommand{\VereinsnameKurz}{MInfo e.V.}
\newcommand{\DatumSatzung}{9. Juni 2018}

% Definition: Titelseite
\title{Satzung des \Vereinsname{}}
\subtitle{Originalfassung}
\author{}
\date{Beschlossen auf der Gründungsversammlung am \DatumSatzung{} in \Stadt{}}

% Definition der PDF-Metadaten
\hypersetup{
	pdftitle={Satzung des \Vereinsname{}},
	pdfsubject={Satzung},
    pdfauthor={\Vereinsname{}},
	pdfkeywords={Satzung, \VereinsnameKurz{}, \Vereinsname{}},
}

\usepackage{draftwatermark}
\SetWatermarkLightness{0.95}
\SetWatermarkScale{2}
\SetWatermarkText{v0.5}
\subtitle{Entwurf Version 0.5}

\begin{document}
\pagenumbering{roman}

\clearpage
\phantomsection
\addcontentsline{toc}{chapter}{Titelseite}
\maketitle

\clearpage
\phantomsection
\addcontentsline{toc}{chapter}{Inhaltsverzeichnis}
\tableofcontents

\clearpage
\pagenumbering{arabic}
%!TEX root = ../Satzung.tex
\chapter{Allgemeine Bestimmungen}
\label{sec:Allg_Bestimmung}


\section{Name und Sitz}
	\begin{absätze}
		\item Der Verein führt den Namen \textit{\Vereinsname{}} bzw. in der Kurzform \textit{\VereinsnameKurz{}}.
		\item Der Verein hat seinen Sitz in \Stadt{}.
		\item Er soll in das Vereinsregister eingetragen werden.
	\end{absätze}

\section{Zweck und Aufgaben des Vereins}
	\begin{absätze}
		\item Der Verein verfolgt ausschließlich und unmittelbar gemeinnützige Zwecke im Sinne des Abschnitts "`Steuerbegünstigte Zwecke"' der Abgabenordnung.
		\item Zweck des Vereins ist die Förderung der Erziehung, Volks- und Berufsbildung einschließlich der Studierendenhilfe.
		\item Der Satzungszweck wird insbesondere verwirklicht durch:
		\begin{sätze}
			\item die Förderung der Interessen der Studierenden am \Fachbereich{} der \Uni{},
			\item die ideelle und finanzielle Unterstützung des Fachschaftsrates am \Fachbereich{} gemäß \VerweisFachschaftsrat{} sowie
			\item die Förderung von Lehre und Forschung am \Fachbereich{} der \Uni{}.
		\end{sätze}
	\end{absätze}

\section{Selbstlosigkeit}
	\begin{absätze}
		\item Der Verein ist selbstlos tätig. Er verfolgt nicht in erster Linie eigenwirtschaftliche Zwecke.
		\item Der Verein verwendet seine Mittel nur für die satzungsgemäßen Zwecke.
		\item Die Mitglieder erhalten keine Gewinnanteile und in ihrer Eigenschaft als Mitglieder auch keine sonstigen Zuwendungen aus den Mitteln des Vereins. Sie haben bei ihrem Ausscheiden aus dem Verein oder bei dessen Auflösung keine Ansprüche auf Vereinsvermögen.
		\item Der Verein setzt keine Mittel zur Unterstützung politischer Parteien oder religiöser Gruppen ein.
		\item Der Verein begünstigt keine Person durch Ausgaben, die dem satzungsgemäßen Zweck widersprechen oder durch unverhältnismäßig hohe Vergütungen.
		\item Bei Auflösung des Vereins oder bei Wegfall steuerbegünstigter Zwecke fällt das Vermögen an die Studierendenschaft der \Uni{} -- vertreten durch den Allgemeinen Studierendenausschuss --, der es unmittelbar und ausschließlich für gemeinnützige oder mildtätige steuerbegünstigte, dieser Satzung entsprechende Zwecke zu verwenden hat.
	\end{absätze}

%!TEX root = ../Satzung.tex
\chapter{Rechtsverhältnisse des Vereins und seiner Mitglieder}

\section{Mitgliedschaft}\label{sec:mitgliedschaft}
\begin{absätze}
    \item Ordentliches Mitglied können alle natürlichen Personen werden, die Teil des \Fachbereich{} der \Uni{} sind.
    \item Fördermitglied können alle Personen werden, die die Ziele des Vereins unterstützen.
\end{absätze}

\section{Erwerb der Mitgliedschaft}\label{sec:erwerb_der_mitgliedschaft}
\begin{absätze}
    \item Die Mitgliedschaft wird durch Annahme eines an den Vorstand gerichteten schriftlichen Antrags erworben.
    \item Der Antrag kann durch den Vorstand unter Angabe von Gründen abgelehnt werden. In diesem Fall entscheidet die nächste Mitgliederversammlung über die Aufnahme.
    \item Die Mitgliedschaft kann den Mitgliedern des Fachschaftsrates am \Fachbereich{} gemäß \VerweisFachschaftsrat{} nicht verwehrt werden.
\end{absätze}

\section{Ende der Mitgliedschaft}\label{sec:ende_der_mitgliedschaft}
\begin{absätze}
    \item Die Mitgliedschaft erlischt
    \begin{sätze}
        \item wenn das Mitglied schriftlich gegenüber dem Vorstand seinen Austritt erklärt,
        \item wenn das Mitglied mit der Zahlung der Mitgliedsbeiträge zwei Jahre im Rückstand ist und der Vorstand daraufhin das Ende der Mitgliedschaft feststellt oder
        \item mit dem Tod des Mitglieds.
    \end{sätze}
    \item Die ordentliche Mitgliedschaft geht in eine Fördermitgliedschaft über, wenn
    \begin{sätze}
        \item das Mitglied in einem Zeitraum von zwei Jahren zu keiner Mitgliederversammlung erschienen ist oder
        \item das Mitglied nicht mehr die Bedingungen des \ref{sec:mitgliedschaft} Abs. 1 erfüllt. Das Mitglied muss die Nichterfüllung dem Vorstand anzeigen.
    \end{sätze}
    \item Ein Mitglied kann aus dem Verein ausgeschlossen werden, wenn das Mitglied
    \begin{sätze}
        \item gegen die Satzung verstößt,
        \item das Vereinsansehen schädigt oder
        \item den Verein vorsätzlich oder fahrlässig wirtschaftlich in erheblichem Ausmaß schädigt
    \end{sätze}
    und die Mitgliederversammlung daraufhin mit einer Mehrheit von mindestens drei Vierteln den Ausschluss beschließt. Das auszuschließende Mitglied wird gesondert eingeladen und bei seinem Erscheinen angehört. Bei der Abstimmung ist das betroffene Mitglied nicht stimmberechtigt.
	\end{absätze}

\section{Mitgliedsbeiträge}
	\begin{absätze}
		\item Von den ordentlichen Mitgliedern und den Fördermitgliedern kann ein jährlicher, ggf. nach Mitgliedsform differenzierter Beitrag erhoben werden, welcher unabhängig vom Beitrittstermin jeweils für das Kalenderjahr erhoben wird.
		\item Über die Höhe der Beiträge entscheidet die Mitgliederversammlung.
	\end{absätze}

\section{Organe des Vereins}
    Organe des Vereins sind:
	\begin{compactenum}
		\item Mitgliederversammlung und
		\item der Vorstand.
	\end{compactenum}

%!TEX root = ../Satzung.tex
\chapter{Mitgliederversammlung}

\section{Zuständigkeiten}
\begin{absätze}
	\item Oberstes Organ des Vereins ist die Mitgliederversammlung, die satzungsgemäß einberufene Versammlung der Vereinsmitglieder.
    \item Der Mitgliederversammlung obliegt die Ordnung der Angelegenheiten des Vereins durch Beschlussfassung. Diese sind unter anderem:
    \begin{sätze}
        \item Wahl und Abwahl des Vorstandes,
        \item Entlastung des Vorstands,
        \item Wahl der Kassenprüfer*innen des Vereins,
        \item Entscheidung über die Erhebung von Beiträgen und Verabschiedung einer Beitragsordnung,
        \item Annahme von durch den Vorstand abgelehnten Mitgliedsanträgen gemäß \ref{sec:erwerb_der_mitgliedschaft} Abs. 2,
        \item Entscheidung über Ausschluss von Mitgliedern gemäß \ref{sec:ende_der_mitgliedschaft} Abs. 3,
        \item Änderungen der Vereinssatzung sowie
        \item Auflösung des Vereins.
    \end{sätze}
\end{absätze}

\section{Turnus, Öffentlichkeit}
\begin{absätze}
    \item Die ordentliche Mitgliederversammlung findet jährlich statt.
    \item Jede Mitgliederversammlung findet öffentlich statt.
\end{absätze}

\section{Einberufung}\label{sec:einberufung}
Die Einladung zur Mitgliederversammlung wird den Mitgliedern spätestens vierzehn Tage vor der Mitgliederversammlung unter Angabe der Tagesordnung an die von ihnen angegebene E-Mail-Adresse vom Vorstand zugesandt.

\section{Tagesordnung}
\begin{absätze}
    \item Die Mitglieder können bis zum Beginn einer Mitgliederversammlung die Ergänzung der Tagesordnung bei dem Vorstand beantragen. Über die Annahme der Anträge entscheidet die Mitgliederversammlung. Nicht Gegenstand eines Ergänzungsantrags können sein:
    \begin{sätze}
        \item Auflösung des Vereins,
        \item Änderung der Satzung sowie
        \item Ausschluss eines oder mehrerer Mitglieder.
    \end{sätze}
    \item Einzelne Tagesordnungspunkte können auf Beschluss der Mitgliederversammlung unter Ausschluss der Öffentlichkeit behandelt werden.
    \item Die Tagesordnung jeder ordentlichen Mitgliederversammlung muss mindestens die folgenden Tagesordnungspunkte beinhalten:
    \begin{sätze}
        \item Beschluss der Tagesordnung,
        \item Feststellung der Beschlussfähigkeit,
        \item Annahme des Protokolls der letzten Mitgliederversammlung,
        \item Geschäftsbericht des Vorstandes,
        \item Bericht der Kassenprüfer*innen sowie
        \item Wahl und Entlastung des Vorstandes.
    \end{sätze}
\end{absätze}

\section{Beschlussfähigkeit}
\begin{absätze}
    \item Eine ordentliche oder außerordentliche Mitgliederversammlung ist beschlussfähig, wenn sie gemäß \ref{sec:einberufung} Abs. 2 einberufen wurde und mindestens die Hälfte der ordentlichen Mitglieder anwesend sind.
    \item Falls die Mitgliederversammlung ihre eigene Beschlussunfähigkeit feststellt, muss der Vorstand binnen sieben Tagen zu einer weiteren Mitgliederversammlung mit gleicher Tagesordnung einladen. Die Einladung muss unter Berücksichtigung des \ref{sec:einberufung} erfolgen. Diese zweite Mitgliederversammlung ist unabhängig von der Zahl der anwesenden, ordentlichen Mitglieder beschlussfähig, sofern dies auf der Einladung angegeben ist.
\end{absätze}

\section{Leitung der Mitgliederversammlung}
Die Leitung der Mitgliederversammlung wird vom Vorstand festgelegt.

\section{Stimmrecht}
Jedes anwesende, ordentliche Mitglied ist in der Mitgliederversammlung einfach stimmberechtigt. Fördermitglieder sind nicht stimmberechtigt.

\section{Beschlussfassung}
\begin{absätze}
    \item Bei Beschlussfassungen kann jedes anwesende, stimmberechtigte Mitglied mit \textit{Ja}, \textit{Nein} oder \textit{Enthaltung} stimmen. Zur Beschlussfassung genügt, falls nicht anders geregelt, eine Mehrheit der \textit{Ja}- über den \textit{Nein}-Stimmen. Bei Stimmengleichheit ist der Antrag abgelehnt.
    \item Bei der Wahl des Vorstandes oder der Kassenprüfer*innen werden, falls mehr Kandidat*innen zur Wahl stehen, als es Ämter zu besetzen gilt, die drei bzw. zwei Kandidat*innen, die die meisten Stimmen auf sich vereinigen können, gewählt. Jedes abstimmende Mitglied hat so viele Stimmen, wie Ämter zu besetzen sind.
    \item Abstimmungen erfolgen stets geheim. Davon kann abgewichen werden, falls kein stimmberechtigtes Mitglied Einspruch erhebt, jedoch niemals, wenn Gegenstand der Abstimmung die Besetzung eines Amtes ist.
    \item Das Nähere kann eine Geschäftsordnung regeln.
\end{absätze}

\section{Protokoll}
\begin{absätze}
    \item Ein vom Vorstand benanntes Vereinsmitglied fertigt ein Protokoll der Mitgliederversammlung an.
    \item Das Protokoll gibt Aufschluss über die Ergebnisse von Abstimmungen.
    \item Das Protokoll ist binnen sieben Tagen den Mitgliedern bekannt zu machen.
    \item Den Vereinsmitgliedern bleiben dreißig Tage nach der Bekanntmachung des Protokolls zum Stellen von Änderungsanträgen, über die der Vorstand entscheidet. Wird das Protokoll dadurch nachträglich geändert, muss das Protokoll erneut entsprechend obiger Bestimmungen bekannt gemacht werden.
\end{absätze}

\section{Außerordentliche Mitgliederversammlung}
\begin{absätze}
    \item Der Vorstand veranstaltet eine außerordentliche Mitgliederversammlung, wenn
    \begin{sätze}
        \item die Belange des Vereins es erfordern,
        \item die Kassenprüfer*innen es für erforderlich halten oder
        \item mindestens ein Drittel der ordentlichen Mitglieder gemeinsam unter Angabe der Tagesordnung bei dem Vorstand eine solche beantragen.
	\end{sätze}
\item Die außerordentliche Mitgliederversammlung findet zum frühsten möglichen Zeitpunkt statt. Die in \ref{sec:einberufung} bestimmten Vorschriften zur Einberufung sind davon unberührt.
\end{absätze}

%!TEX root = ../Satzung.tex
\chapter{Vorstand}

\section{Zusammensetzung des Vorstandes}\label{cha:vorstand}
\begin{absätze}
    \item Der Vorstand besteht aus drei gleichberechtigten Personen.
    \item Zusätzlich können bis zu vier Beisitzer*innen gewählt werden.
    \item Eines der Vorstandsmitglieder gemäß Abs. 1 muss dabei zusätzlich das Amt des*der Kassenbeauftragten bekleiden und ist mit der Führung der Kasse beauftragt.        \end{absätze}

\section{Rechtsvertretung des Vereins}
\begin{absätze}
    \item Vorstand im Sinne des § 26 BGB sind die Vorstandsmitglieder gemäß \ref{cha:vorstand} Abs. 1. Sie vertreten den Verein gemeinsam.
\end{absätze}

\section{Ehrenamtlichkeit des Vorstandes}
Die Vorstandsmitglieder sind ehrenamtlich tätig. Entstehende finanzielle Aufwendungen können ihnen vom Verein erstattet werden.
    
\section{Aufgaben des Vorstandes}
\begin{absätze}
    \item Der Vorstand führt die Geschäfte des Vereins im Rahmen dieser Satzung.
    \item Der Vorstand führt die Beschlüsse der Mitgliederversammlung aus.
    \item Der Vorstand hat auf jeder Mitgliederversammlung Rechenschaft über die seit der letzten Mitgliederversammlung geleistete Arbeit abzulegen.
\end{absätze}

\section{Bestellung und Abberufung des Vorstandes}
\begin{absätze}
    \item Der Vorstand wird von der Mitgliederversammlung auf die Dauer eines Jahres gewählt. Dabei wird auch die Rolle des*der Kassenbeauftragten festgelegt. Wiederwahl ist zulässig.
    \item Jedes Mitglied des Vorstands muss zum Zeitpunkt der Wahl ordentliches Mitglied des Vereins sein.
    \item Nach Ablauf der Amtszeit bleiben die Mitglieder des Vorstands bis zu einer Neuwahl des Vorstands im Amt.
    \item Eine vorgezogene Neuwahl des Vorstands kann von der Mitgliederversammlung mit einer absoluten Mehrheit beschlossen werden.
\end{absätze}

\section{Vorstandssitzungen}
\begin{absätze}
    \item Vorstandssitzungen finden nach Bedarf statt und werden grundsätzlich einvernehmlich einberufen. In Ausnahmefällen ist eine Vorstandssitzung aber auch dann einzuberufen, wenn dies von mindestens der Hälfte der Mitglieder des Vorstands beantragt wird.
    \item Der Vorstand ist beschlussfähig, wenn mindestens die Hälfte der Vorstandsmitglieder anwesend sind. Die Beschlüsse des Vorstands werden mit Stimmenmehrheit der anwesenden Vorstandsmitglieder gefasst.
    \item Das Nähere kann eine Geschäftsordnung regeln.
\end{absätze}

%!TEX root = ../Satzung.tex
\chapter{Verschiedenes}

\section{Rechnungslegung, Kassenprüfung}
\begin{absätze}
	\item Der Vorstand hat der Mitgliederversammlung über die Kassenführung Rechnung zu legen und fertigt einen Jahresabschluss an.
	\item Mit dem Vorstand wählt jede Mitgliederversammlung für die gleiche Amtszeit zwei Vereinsmitglieder zu Kassenprüfer*innen. Sie prüfen zum Ende jedes Geschäftsjahres die Kassenführung und den Jahresabschluss auf Richtigkeit und Vollständigkeit. Sie berichten der Mitgliederversammlung und beantragen gegebenenfalls die Entlastung des Vorstandes.
	\item Die Kassenprüfer*innen dürfen nicht Mitglieder des Vorstandes sein. Falls andere Kandidaten zur Verfügung stehen, dürfen sie nicht Mitglieder des Vorstandes des Vorjahres sein und nicht wiedergewählt werden.
\end{absätze}

\section{Schlussbestimmung}
Die Satzung ist von der Mitgliederversammlung am \DatumSatzung{} beschlossen und tritt am gleichen Tag in Kraft.

\vfill % Platz für Unterschriften

\Stadt{}, den \DatumSatzung{}

Die Gründungsmitglieder


\end{document}
